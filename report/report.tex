\documentclass[a4paper, 14pt]{article}


\usepackage{cmap}
\usepackage[T2A]{fontenc}
\usepackage[utf8]{inputenc}
\usepackage[english,russian]{babel}
\usepackage{amsmath}
\usepackage{amsfonts}
\usepackage{amsmath,amsthm,amssymb}
\usepackage{color}
\usepackage{pgfplots}
\usepackage{tikz}
\pgfplotsset{compat=1.9}
\usepackage{graphicx}
\usepackage{float}
\usepackage{wrapfig}
\usepackage{bchart}




\begin{document}

	\renewcommand{\chaptername}{Лабораторная работа}
	\def\contentsname{Содержание}

	\begin{titlepage}
		\begin{center}
			\textsc{<<НАЦИОНАЛЬНЫЙ ИССЛЕДВАТЕЛЬСКИЙ УНИВЕРСИТЕТ ИТМО">\\[5mm]
			Факультет информационных технологий и программирования\\[2mm]
			Кафедра компьютерных технологий}

			\vfill

			\textbf{ОТЧЁТ ПО ЛАБОРАТОРНОЙ РАБОТЕ №4\\[3mm]
			Изучение алгоритмов метода Ньютона и его модификаций, в том числе квазиньютоновских методов.
\\[28mm]
			}
		\end{center}

		\hfill
		\begin{minipage}{.5\textwidth}
			Выполнили студенты:\\[2mm]
			Ефимов Сергей Алексеевич\\
			группа: М3237\\[2mm]
			Соколов Александр Андреевич\\
			группа: М3234\\[5mm]

			Проверил:\\[2mm]
			Свинцов Михаил Викторович
		\end{minipage}%
		\vfill
		\begin{center}
			г. Санкт-Петербург
		\end{center}
	\end{titlepage}


	\section*{Постановка задачи}
	Задача лаборатрной работы  -- научиться реализовывать алгоритмы градиентной оптимизации каждым из следуших способов:
	\begin{enumerate}
		\item Метод градиентного спуска
		\item Метод наискорейшего спуска
		\item Метод сопряженных градиентов
	\end{enumerate}
	Также необходимо оценить как меняется скорость сходимости, если для поиска величины шага использовать различные методы одномерного поиска.


\end{document}
